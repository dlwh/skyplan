%File: formatting-instruction.tex
\documentclass[letterpaper]{article}
\usepackage{aaai}
\usepackage{times}
\usepackage{helvet}
\usepackage{courier}
\usepackage{amsmath}
\usepackage{amssymb}
\usepackage{amsfonts}
\usepackage{algorithm}
\usepackage{algorithmicx}
\usepackage{algpseudocode}
\usepackage{amsthm}

\theoremstyle{plain} \newtheorem{theorem}{Theorem} \newtheorem{proposition}{Proposition} \newtheorem{lemma}{Lemma}
\newtheorem*{corollary}{Corollary}  \newtheorem{claim}{Claim} 

\theoremstyle{definition} \newtheorem{definition}{Definition} \newtheorem{conjecture}{Conjecture} \newtheorem*{example}{Example} 

\theoremstyle{remark} \newtheorem*{remark}{Remark} \newtheorem*{note}{Note} \newtheorem{case}{Case}


\frenchspacing
\pdfinfo{
/Title (Faster Optimal Planning with Partial-Order Pruning)
/Subject (AAAI Publications)
/Author (Redacted)}
\setcounter{secnumdepth}{0}  
 \begin{document}
% The file aaai.sty is the style file for AAAI Press 
% proceedings, working notes, and technical reports.
%
\title{Faster Optimal Planning with Partial-Order Pruning}
%%\author{AAAI Press\\
%Association for the Advancement of Artificial Intelligence\\
%2275 East Bayshore Road, Suite 160\\
%Palo Alto, California 94303\\
%}

\maketitle
\begin{abstract}
\begin{quote}
  When planning problems have many kinds of resources or high
  concurrency, each optimal state has exponentially many minor
  variants, some of which are "better" than others. Standard methods
  like A* cannot effectively exploit these minor relative differences,
  and therefore must explore many redundant, clearly suboptimal
  plans. We describe a new optimal search algorithm for planning
  that leverages a partial order relation between states. Under
  suitable conditions, states that are dominated by another with
  respect to this order can be pruned while provably maintaining
  optimality. We also describe a simple method for automatically
  discovering compatible partial orders in both serial and concurrent
  domains. In our experiments we find that more than 99\% of search
  states can be pruned in some domains.
\end{quote}
\end{abstract}

\section{Introduction}

Planning problems differ from other search problems in a number of
ways. One important distinction is that there are typically many
planning states that are ``similar'' along one or more dimensions.
For instance, in a job shop scheduling domain, one search state
might have the same number of widgets---but fewer sprockets---than
another. Assuming that more is always better and that the two states
can be reached in the same time, we can safely discard the former,
\textit{dominated} states, instead focusing our search on the better
state.

In this paper, we seek to formalize this notion, using Pareto
dominance to prune states that are strictly dominated by some other
states. More specifically, we give conditions under which we can
expand only those states in the \textit{skyline}, that is, states
that are not dominated by any another state. Our system, Skyplan,
is a refinement of Uniform Cost Search or A$^*$ that expands only
those states that are on the skyline.

The central idea underlying our approach is to define a partial
order relationship between states in the search space. This partial
order has an intuitive interpretation: one state dominates another
if it has no fewer ``good'' resources (e.g. jobshop outputs) than
another, no more ``bad'' resources (e.g. labor expended or time
taken), and it is better in one or more ways. 

In addition to proving the correctness of our approach, we also show 
how to automatically infer such a partial order from a problem
specification such as PDDL. (XXX cite?)  XXX


XXX if we do this:
Moreover, we describe how to incorporate non-optimal, heuristic
plans into Skyplan by not expanding states.  We believe this is the
first search procedure that can safely use information from non-optimal
search algorithms for optimal search. Moreover, we describe how to
convert a combination of a heuristic planner with Skyplan to produce
an anytime search algorithm, similar to that proposed in XXX.

In our experiments, we compare Skyplan to a similar implementation
of $A^*$ on several (XXX) job shop domains. In addition, we introduce
two new domains, based on the popular video game StarCraft. (XXX
we should credit evo chamber for one of them.) Skyplan performs
especially well in these scenarios, cutting the branching factor
by as much as 99\% compared to A$^*$.  Furthermore, we compare our
algorithm to a heuristic open source planner specifically engineered
to one of the StarCraft domains.

\section{Skyplan}

\begin{figure}
  \caption{Pareto Optimality figure}
\end{figure}

We operate in the standard convention of planning as search within
a weighted directed graph.  A planning problem consists of a directed
graph $G$ with states $n$ consisting of planning states.  States are
linked by directed edges with an associated a cost function
$\mathrm{cost}(n_s,n_t) \in \mathbb R^+$.  There is also a privileged
initial state $s$, along with a set of goal states $F$.  Further
define $g(n)$ to be the minimum cost path from $s$ to any state $n$.
Our goal is to find a path from $s$ to some state $n_g \in F$ with
the lowest g cost.


\subsection{Partial Orders}

We will further assume that our graph is endowed with a partial order $\preceq$
that relates states $n$ to one another, with the intuitive semantics that
$n \preceq m$ if $n$ is no better than $m$ in any way. For example, in
our sprockets and widgets example, $n \prec m$ if $n$ has no more widgets and
no more sprockets than $m$ and if $g(n) \ge g(m)$. Strict dominance
holds if $m$ is in addition strictly better than $n$ in some way.

Our goal is to define an optimal search algorithm that can exploit
this partial order to reduce the search space by only expanding
states $n$ that are \textit{weakly Pareto optimal}, that is states
$n$ with $n \nprec n'$ for all states $n'$. For any set of states
$N$, we define the \textit{skyline} of that set as
$\textrm{skyline}(N)=\{n: \ne n' \in N: n \prec n'\}$. That is, the
skyline of a set is those states which are weakly Pareto optimal.
In subsequent sections, we will give a useful sufficient condition
under which we can exploit a partial order while preserving correctness
of optimal graph search algorithms.

We are not the first to suggest the use of Pareto optimality or skyline queries
in the context of planning. XXX By contrast, we are the first to define conditions
under which we can \textit{prune} search states while still maintaining optimality.
XXX some people looked at pruning\dots


\subsection{Algorithm}

Skyplan is a fairly straightforward modifications of Uniform
Cost Search or A$^*$. For the sake of exposition, we focus on Uniform Cost
Search, though A$^*$ or any optimal graph search algorithm can
be modified in the same way.

Skyplan is defined in Algorithm \ref{alg:skyplan}. Essentially, we run Uniform Cost Search as normal, except that 
we only expand states that are not strictly dominated by another
state we've either expanded or enqueued for expansion. That is, we only expand nodes that are in the \textit{skyline}
of the nodes explored so far.

\begin{algorithm}
  \begin{algorithmic}[1]
    \Procedure{Skyplan}{XXX}
    \State Initialize $Open=\{\langle s,0\rangle\}$, $Closed=\{\}$, $Pruned=\{\}$
    \While{ $Open$ is not empty}
    \State Pop the minimum cost state $\mathbf{n} = \langle n,c\rangle\in Open$
      \State $n\rightarrow Closed$
      \If {$n\in F$} 
        \State \Return the path to $n$ following back pointers
      \EndIf
      \For {$n^\prime\in\mathrm{succ}(n)$} 
      \State $\mathbf{n^\prime} \gets \langle n^\prime,c+\mathrm{cost}(n,n^\prime)\rangle$
        \If {$\forall m \in Open \cup Closed. n^\prime \nprec m$}
        \State $\mathbf{n^\prime} \rightarrow Open$
        \EndIf
      \EndFor
    \EndWhile
  \EndProcedure
  \end{algorithmic}
\caption{Skyplan}
\label{alg:skyplan}
\end{algorithm}

\subsection{Compatibility}
\begin{figure}
  \caption{Compatible partial orders. Partial orders stay ahead.}
\end{figure}

Not just any partial order on states can be used to preserve optimality. As a perverse
example, we can define a partial order under which all states on the optimal path
are dominated by non-optimal states. Thus, we need to define a property
under which a partial order is \textit{compatible} with a search graph. While
a broad class of partial orders might work, we have identified a particular 
property that is especially applicable to planning problems:

\begin{definition}[Compatibility]
  A partial order $\prec$ is \textit{compatible} if for all states $n \preceq m$,
  \begin{enumerate}
    \item $g(n) \ge g(m)$ and 
    \item $\forall n' \in \mathrm{succ(n)}$ $\exists m' \in \mathrm{succ}(m')$ such that $n' \preceq m'$.
    \item If $n \in F$, then $m \in F$.
  \end{enumerate}
\end{definition}
This recursive definition of compatibility essentially means that if a state $n$
is dominated by a state $m$, then it must be the case that $n$ is no cheaper than $m$
and that $m$'s successors ``stay ahead'' of $n$'s. Finally, $F$ must be closed with respect to $\succeq$: every state that
dominates a goal state must also be a goal. It is worth mentioning that
there is always a compatible partial ordering: the trivial order with $n \npreceq m$ for all
states $n$ and $m$.

In planning problems, defining a compatible partial order is usually quite easy. For
resources where ``more is better''. In the sequel, we discuss a standard structure
for these partial orders, as well as how to automatically infer a partial order
from the specification of a planning problem.

\subsection{Analysis}

XXX

\begin{claim}{}\label{clm-complete}
   If $\preceq$ is a compatible partial order, Skyplan
is \emph{complete}.
\end{claim}
\begin{proof} Suppose otherwise, then all paths to all goals must
have been pruned. Consider one such path $p$ and let $n$ be the
state in $p$ that was pruned, and let $p_{n:}$ be the suffix of the
path starting from $n$. Let $m$ be a state with $m \succ n$ that was
not pruned.  By transitivity of $\preceq$, such a state must exist.
Then, from condition and (2) and (3) of compatibility, a simple
inductive argument shows that there must be an analogous path 
$p^\prime_{m:}$ ending in a goal state.

  XXX We should say something ensuring that no states in $p^\prime$ are pruned.
\end{proof}
  
\begin{claim}{}\label{clm-optimal}
   If $\preceq$ is a compatible partial order, Skyplan
is \emph{optimal}.
\end{claim}
\begin{proof} Suppose otherwise, then consider an optimal path $p^*$ with final state $f^*$
  must have been pruned. Let $n$ be the state along $p^*$ that
  was pruned, and let $m$ be a non-pruned state with $m \succ n$.
  Then, again from condition (2) and (3), there must be an analogous path
  starting from $m$ and running to a final state $f^\prime$. Moreover, by condition (1)
  we have that the total cost of this path $g(f^\prime) \le g(f^*)$, a contradiction.

  XXX We should say something ensuring that no states in $p^\prime$ are pruned.
\end{proof}



\begin{claim}{}\label{clm-complete}
For any compatible partial order $\prec$, Skyplan is \emph{optimally efficient}.
\end{claim}
\begin{proof} XXX \end{proof}

\section{Inferring Partial Orders}

\section{Exploiting Heuristic Planners}

\section{Experiments}

\subsection{The StarCraft Domain}
\begin{figure}
  \caption{states pruned table?}
\end{figure}
\subsection{The Woodworking Domain}
\subsection{The StarCraft 2 Domain}
\begin{figure}
  \caption{Plan length over time? Anytime plan length over time?}
\end{figure}

XXX stuff about Starcraft 2 and Evo Chamber
XXX http://www.pcgamer.com/2010/11/02/computer-program-finds-devastating-starcraft-2-build-orders/

\section{Conclusion}

\end{document}
