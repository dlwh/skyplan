%File: formatting-instruction.tex
\documentclass[letterpaper]{article}
\usepackage{aaai}
\usepackage{times}
\usepackage{helvet}
\usepackage{courier}
\frenchspacing
\pdfinfo{
/Title (Formatting Instructions for Authors Using LaTeX)
/Subject (AAAI Publications)
/Author (AAAI Press)}
\setcounter{secnumdepth}{0}  
 \begin{document}
% The file aaai.sty is the style file for AAAI Press 
% proceedings, working notes, and technical reports.
%
\title{Optimal Planning using Pareto Optimality}
%%\author{AAAI Press\\
%Association for the Advancement of Artificial Intelligence\\
%2275 East Bayshore Road, Suite 160\\
%Palo Alto, California 94303\\
%}

\maketitle
\begin{abstract}
\begin{quote}
  Planning problems differ from more typical search problems in a
  number of ways.  One important difference is that many states are
  very similar to one another, but not necessarily in a way that
  is easily expressed using a standard heuristic function.  We
  describe a new optimal search algorithm for planning that leverages
  a partial order relation between states. Under suitable conditions,
  states that are dominated by another with respect to this order
  can be pruned while provably maintaining optimality.  Moreover,
  using our algorithm, we show how to incorporate plans found by
  non-optimal heuristic planners while still maintaining optimality.
  We describe how to automatically discover these compatible partial
  orders in serial and concurrent domains.  In our experiments we
  find that more than 99\% of search nodes can be pruned in some
  domains.
\end{quote}
\end{abstract}

\section{Introduction}

Compared to other search problems commonly considered in artificial
intelligence, planning is characterized by states that are ``similar''
along one or more dimensions.  For instance, in a job shop domain,
one search state might have more widgets---but fewer sprockets---than
another. On the other hand, there might be another state that has
more widgets and sprockets than both states in a shorter amount of
time. Assuming that more is always better, we can safely discard
these two \textit{dominated} states, instead focusing our search
on the better state.

In this paper, we seek to formalize this notion, using Pareto
dominance to prune states that are strictly dominated by some other
states. More specifically, we give conditions under which we need
only consider those states that are not dominated by any another
state, also known as the \textit{skyline}. Our system, Skyplan, is
a refinement of A$^*$ that expands only those states that are on
the skyline. (XXX first and second sentence of this para are kinda
redundant.)

XXX somewhere in here, we need to say why we can't put this in the heuristic.

Moreover, we describe how to incorporate non-optimal, heuristic
plans into Skyplan by not expanding states.  We believe this is the
first search procedure that can safely use information from non-optimal
search algorithms for optimal search. Moreover, we describe how to
convert a combination of a heuristic planner with Skyplan to produce
an anytime search algorithm, similar to that proposed in XXX.

In our experiments, we compare Skyplan to a similar implementation of $A^*$
in domains based on real time strategy games. Furthermore, we compare
our algorithm to an heuristic open source planner. XXX

\section{Skyplan}

\begin{figure}
  \caption{Pareto Optimality figure}
\end{figure}

\subsection{Partial Orders}

We are not the first to suggest the use of Pareto optimality or skyline queries
in the context of planning. XXX By contrast, we are the first to define conditions
under which we can \textit{prune} search states while still maintaining optimality.

\subsection{Compatibility}
\begin{figure}
  \caption{Compatible partial orders. Partial orders stay ahead.}
\end{figure}

\subsection{Correctness}

\section{Inferring Partial Orders}

\section{Exploiting Heuristic Planners}

\section{Experiments}

\subsection{The StarCraft Domain}
\begin{figure}
  \caption{Nodes pruned table?}
\end{figure}
\subsection{The Settlers Domain}
\subsection{The StarCraft 2 Domain}
\begin{figure}
  \caption{Plan length over time? Anytime plan length over time?}
\end{figure}

XXX stuff about Starcraft 2 and Evo Chamber
XXX http://www.pcgamer.com/2010/11/02/computer-program-finds-devastating-starcraft-2-build-orders/

\section{Conclusion}

\end{document}
