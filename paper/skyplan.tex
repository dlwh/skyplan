%File: formatting-instruction.tex
\documentclass[letterpaper]{article}
\usepackage{aaai}
\usepackage{times}
\usepackage{helvet}
\usepackage{courier}
\usepackage{amsmath}
\usepackage{amssymb}
\usepackage{amsfonts}
\usepackage{algorithm}
\usepackage{amsthm}
\theoremstyle{plain} \newtheorem{theorem}{Theorem} \newtheorem{proposition}{Proposition} \newtheorem{lemma}{Lemma} \newtheorem*{corollary}{Corollary}

\theoremstyle{definition} \newtheorem{definition}{Definition} \newtheorem{conjecture}{Conjecture} \newtheorem*{example}{Example} 

\theoremstyle{remark} \newtheorem*{remark}{Remark} \newtheorem*{note}{Note} \newtheorem{case}{Case}


\frenchspacing
\pdfinfo{
/Title (Formatting Instructions for Authors Using LaTeX)
/Subject (AAAI Publications)
/Author (AAAI Press)}
\setcounter{secnumdepth}{0}  
 \begin{document}
% The file aaai.sty is the style file for AAAI Press 
% proceedings, working notes, and technical reports.
%
\title{Optimal Planning using Pareto Optimality}
%%\author{AAAI Press\\
%Association for the Advancement of Artificial Intelligence\\
%2275 East Bayshore Road, Suite 160\\
%Palo Alto, California 94303\\
%}

\maketitle
\begin{abstract}
\begin{quote}
  Planning problems differ from more typical search problems in a
  number of ways.  One important difference is that many states are
  very similar to one another, but not necessarily in a way that
  is easily expressed using a standard heuristic function.  We
  describe a new optimal search algorithm for planning that leverages
  a partial order relation between states. Under suitable conditions,
  states that are dominated by another with respect to this order
  can be pruned while provably maintaining optimality.  Moreover,
  using our algorithm, we show how to incorporate plans found by
  non-optimal heuristic planners while still maintaining optimality.
  We describe how to automatically discover these compatible partial
  orders in serial and concurrent domains.  In our experiments we
  find that more than 99\% of search nodes can be pruned in some
  domains.
\end{quote}
\end{abstract}

\section{Introduction}

Compared to other search problems commonly considered in artificial
intelligence, planning is characterized by states that are ``similar''
along one or more dimensions.  For instance, in a job shop domain,
one search state might have more widgets---but fewer sprockets---than
another. On the other hand, there might be another state that has
more widgets and sprockets than both states in a shorter amount of
time. Assuming that more is always better, we can safely discard
these two \textit{dominated} states, instead focusing our search
on the better state.

In this paper, we seek to formalize this notion, using Pareto
dominance to prune states that are strictly dominated by some other
states. More specifically, we give conditions under which we need
only consider those states that are not dominated by any another
state, also known as the \textit{skyline}. Our system, Skyplan, is
a refinement of A$^*$ that expands only those states that are on
the skyline. (XXX first and second sentence of this para are kinda
redundant.)

XXX somewhere in here, we need to say why we can't put this in the heuristic.

Moreover, we describe how to incorporate non-optimal, heuristic
plans into Skyplan by not expanding states.  We believe this is the
first search procedure that can safely use information from non-optimal
search algorithms for optimal search. Moreover, we describe how to
convert a combination of a heuristic planner with Skyplan to produce
an anytime search algorithm, similar to that proposed in XXX.

In our experiments, we compare Skyplan to a similar implementation of $A^*$
in domains based on real time strategy games. Furthermore, we compare
our algorithm to an heuristic open source planner. XXX

\section{Skyplan}

\begin{figure}
  \caption{Pareto Optimality figure}
\end{figure}

We operate in the standard convention of planning as search within a weighted directed graph. Let
A planning problem consists of a directed graph $G$ with nodes $n$ consisting of planning states,
states are linked by directed edges $a$ corresponding to actions within plan space. We associate
a cost function $\mathrm{cost}(n_s,a,n_t) \in \mathbb R^+$ for taking action $a$ in state $n_s$ yielding a result $n_t$.
XXX There is also a privileged initial state $n_i$, along with a set of goal states $F$. 
Further define $g(n)$ to be the minimum cost path from $n_i$ to any node $n$. Our
goal is to find a path from $n_i$ to some state $n_g \in F$ with the lowest g cost..


\subsection{Partial Orders}



We will further assume that our graph is endowed with a partial order $\prec$
that relates nodes $n$ to one another, with the intuitive semantics that
$n \prec m$ if $n$ is no better than $m$ in any way. For example, in
our sprockets and widgets example, $n \prec m$ if $n$ has no more widgets and
no more sprockets than $m$ and if $g(n) \ge g(m)$. 

Our goal is to define an optimal search algorithm that can exploit
this partial order to reduce the search space significantly by only
expanding nodes $n$ that are \textit{weakly Pareto optimal}, that
is nodes $n$ with $n \nprec n'$ for all nodes $n'$. For any set of
nodes $N$, we define the \textit{skyline} of that set as
$\textrm{skyline}(N)=\{n: \ne n' \in N: n \prec n'\}$. That is, the
skyline of a set is those nodes which are weakly Pareto optimal.
In subsequent sections, we will give a useful sufficient condition under
which we can exploit a partial order while preserving correctness of optimal graph
search algorithms.

We are not the first to suggest the use of Pareto optimality or skyline queries
in the context of planning. XXX By contrast, we are the first to define conditions
under which we can \textit{prune} search states while still maintaining optimality.


\subsection{Algorithm}

Skyplan is a fairly straightforward modifications of Uniform
Cost Search or A$^*$. For the sake of exposition, we focus on Uniform Cost
Search, though A$^*$ or any optimal graph search algorithm can
be modified in the same way.

Skyplan is defined in Algorithm \ref{alg:skyplan}. Essentially, we run Uniform Cost Search as normal, except that 
we only expand nodes that are not strictly dominated by another
node we've either expanded or enqueued for expansion.

\begin{algorithm}
\begin{enumerate}
\item Initialize: $Open=\{\langle s,0\rangle\}$, $Closed=\{\}$,
$Pruned=\{\}$
\item Pop the min-cost path $\mathbf{n}\in Open$, $\mathbf{n}\rightarrow Closed$

\begin{enumerate}
\item If $n\in T$: return $\langle n,prev(n),\hat{h}(s,n)\rangle$
\end{enumerate}
\item $\forall n\prime\in\Gamma(n)$, let $\mathbf{n\prime}=\langle n\prime,n,\hat{h}(s,n)+cost(n,n\prime)\rangle$:

\begin{enumerate}
\item If $\exists\mathbf{m}\in Open\cup Closed\,:\,\mathbf{m}\succeq\mathbf{n\prime}$:
$\mathbf{n\prime}\rightarrow Pruned$
\item Else: 

\begin{enumerate}
\item Let $Prunable(\mathbf{n\prime})=\{\mathbf{m}\in Open\,:\,\mathbf{n\prime}\succeq\mathbf{m}\}$
\item $Open=Open-Prunable(\mathbf{n\prime})$
\item $\mathbf{n\prime}\rightarrow Open$ 
\end{enumerate}
\end{enumerate}
\end{enumerate}
\caption{Skyplan}
\label{alg:skyplan}
\end{algorithm}

\subsection{Compatibility}
\begin{figure}
  \caption{Compatible partial orders. Partial orders stay ahead.}
\end{figure}

Not just any partial order on nodes can be used to preserve optimality. As a perverse
example, we can define a partial order under which all nodes on the optimal path
are dominated by non-optimal nodes. Thus, we need to define a property
under which a partial order is \textit{compatible} with a search graph. While
a broad class of partial orders might work, we have identified a particular 
property that is especially applicable to planning problems:

\begin{definition}[Compatibility]
  A partial order $\prec$ is \textit{compatible} if for all nodes $n \preceq m$,
  \begin{enumerate}
    \item $g(n) \ge g(m)$ and 
    \item $\forall n' \in \mathrm{succ(n)}$ $\exists m' \in \mathrm{succ}(m')$ such that $n' \preceq m'$.
  \end{enumerate}
\end{definition}
This recursive definition of compatibility essentially means that if a node $n$
is dominated by a node $m$, then it must be the case that $n$ is no cheaper than $m$
and that $m$'s successors ``stay ahead'' of $n$'s. It is worth mentioning that
there is always a compatible partial ordering: the trivial order with $n \npreceq m$ for all
nodes $n$ and $m$.

In planning problems, defining a compatible partial order is usually quite easy. For
resources where ``more is better''. In the sequel, we discuss a standard structure
for these partial orders, as well as how to automatically infer a partial order
from the specification of a planning problem.




The task is to find the shortest path from $s$ to a goal node in
$T$. We use Uniform Cost Search with the addition of pruning. Three
sets $Open$, $Closed$, $Pruned$ (not used in algorithm, just proof)
whose elements are triples $\mathbf{n=}\langle n,\, prev(n),\,\hat{h}(s,n)\rangle$
where $\hat{h}(s,n)$ is our current best guess for the minimum cost
from $s$ to $n$. By following the prev pointers (through Open and
Closed), we can reconstruct a path from $s$ to $n$. Thus, we can
think of these triples as paths themselves. By the operation $\langle n,prev(n),\hat{h}(s,n)\rangle\rightarrow Open$,
(resp. Closed and Pruned), we mean that we replace/update any element
of $\langle n,*,*\rangle\in Open$ with $\langle n,prev(n),\hat{h}(s,n)\rangle$,
adding it if no such element exists.

\subsection{Analysis}

\textbf{Theorem:} If $\succeq$ is a compatible partial order, S{*}
is \emph{complete }and \emph{optimal}.

\textbf{Proof:} Can we appeal to the fact that we are essentially
modifying UCS? In that case, we must simply show that we never prune
an optimal path to the goal (or, more accurately, there is an optimal
path to the goal which we do not prune).

\textbf{Theorem:} S{*} is \emph{optimally efficient}.

\textbf{Proof:} I haven't even thought about if this is true or not.

\section{Inferring Partial Orders}

\section{Exploiting Heuristic Planners}

\section{Experiments}

\subsection{The StarCraft Domain}
\begin{figure}
  \caption{Nodes pruned table?}
\end{figure}
\subsection{The Settlers Domain}
\subsection{The StarCraft 2 Domain}
\begin{figure}
  \caption{Plan length over time? Anytime plan length over time?}
\end{figure}

XXX stuff about Starcraft 2 and Evo Chamber
XXX http://www.pcgamer.com/2010/11/02/computer-program-finds-devastating-starcraft-2-build-orders/

\section{Conclusion}

\end{document}
